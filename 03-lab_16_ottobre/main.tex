\documentclass[a4paper,12pt]{article}
\usepackage{graphicx}       %LaTeX package to import graphics

\usepackage[T1]{fontenc}
\usepackage[italian]{babel}

\usepackage{subcaption}

\usepackage{hyphenat}
\usepackage{array}
\usepackage{booktabs}       % Per linee orizzontali migliori
\usepackage{caption}        % Per personalizzare le didascalie<
\usepackage{multirow}       % Per combinare celle nelle colonne
\usepackage{float}
\usepackage{hyperref}

\usepackage{bm} 

\usepackage{amsmath}        % Per migliorare l'aspetto delle formule

\usepackage{todonotes}      % mettere le note dentro il documento

\usepackage{siunitx}        % Per formattare le unità di misura
\usepackage{gensymb}        % Simboli come °
\usepackage{xfrac}          % per fare le frazioni inclinate

\usepackage{pdfpages}

\usepackage{import}
\usepackage{frontespizio}

\usepackage{placeins}       %per non far andare le immagini al di fuori delle sezioni utilizzare il comando: \FloatBarrier non far superare le immagini quel punto

\usepackage{adjustbox}      % per dimensionare le immagini in modo automatico


\begin{document}

\includepdf{./frontespizio/frontespizio.pdf}

\section*{Trigger di Schmitt}
L'amplificatore operazionale è un circuito elettronico che permette di confrontare due tensioni in ingresso e fornirne la differenza tra le due moltiplicata per un fattore di amplificazione $A$.
\begin{align*}
	V_{out} = A \cdot (V^+ - V^-)
\end{align*}
Nel caso ideale $ A \rightarrow \infty $, pertanto l'uscita $V_{out}$ saturerà alla tensione di alimentazione positiva $V_{DD}$ solo se la differenza tra le due tensioni è maggiore di zero, altrimenti $V_{out} = -V_{DD}$.

Questo permette di utilizzarlo come comparatore di due tensioni. Tuttavia nella realtà sono presente delle problematiche dovute alla presenza di rumore elettronico che porta il segnale in uscita ad avere degli scatti spuri dovuti al ripetuto passaggio della soglia a causa del rumore stesso.
Per risolvere questo problema si utilizza il trigger di Schmitt \ref{fig:trigger_schmitt}.

\begin{figure}[h]
	\centering
	\includegraphics[width = 0.55\linewidth]{./immagini/schmitt/circuito.png}
	\caption{Schematico del Trigger di Schmitt}
	\label{fig:trigger_schmitt}
\end{figure}

Grazie all'utilizzo di una soglia dinamica, dipendente da $V_{out}$, permette di avere in uscita un segnale meno sensibile al rumore in ingresso. In particolare la soglia si modifica secondo questi punti:
\begin{enumerate}
	\item Ipotizzando uno stato iniziale in cui $V_{in} \ll 0$ allora si ha che $V_{out}$ satura a $V_{DD}$, questo porta ad avere un potenziale in $V^+$ pari a:
	      \begin{align*}
		      V^+ = \frac{R_1}{R_2 + R_1} \cdot V_{DD} \overset{\tiny\mathrm{Hp:}\,R_1 = R_2}{\longrightarrow} \frac{V_{DD}}{2} = V^+_H
	      \end{align*}
	      determinando la soglia;
	\item Quando $V_{in} > V^+_H$ allora $V_{out}$ diventa pari a $V_{SS}$, ciò comporta ad avere $V^+ = \frac{V_{SS}}{2} = V^+_L$, ovvero una nuova soglia;
	\item Si rimane nello stato $2$ fino a quando $V_{in}$ diventa inferiore di $V^+_L$, se si supera la soglia si riparte dal punto $1$.
\end{enumerate}
Questo andamento genera un ciclo di isteresi con ampiezza pari a:
\begin{align*}
	\mathrm{Ampiezza}=\frac{2R_1}{R1+R2} \cdot V_{DD} \overset{\tiny\mathrm{Hp:}\,R_1 = R_2}{\longrightarrow} V_{DD}
\end{align*}
La quale si può verificare a figura \ref{fig:schmitt_mod_xy}.

\begin{figure}[h]
	\centering
	\includegraphics[width = 0.7\linewidth]{immagini/schmitt/schmitt_vin_vout_xy.png}
	\caption{Ciclo di isteresi del Trigger di Schmitt}
	\label{fig:schmitt_mod_xy}
\end{figure}

I valori utilizzati per testarne il funzionamento sono riportati a tabella \ref{tab:valori_trigger_schmitt}:
\begin{table}[h]
	\centering
	\setlength{\tabcolsep}{20pt}
	\begin{tabular}{c c}
		\toprule
		Elemento     & Valore            \\
		\midrule
		$V_{DD}$     & $10\,\mathrm{V}$  \\
		$V_{in\,pp}$ & $20\,\mathrm{V}$  \\
		$freq$       & $1\,\mathrm{KHz}$ \\
		$R_1$        & $9.0\,K\Omega$    \\
		$R_2$        & $9.0\,K\Omega$    \\
		\bottomrule
	\end{tabular}
	\caption{Valori utilizzati nel circuitio: Trigger di Schmitt.}
	\label{tab:valori_trigger_schmitt}
\end{table}

Da cui sono stati ricavati i grafici a figura \ref{fig:schmitt_oscilloscopio}.


\begin{figure}[h]
	\centering
	\begin{subfigure}{0.49\linewidth}
		\includegraphics[width = \linewidth]{immagini/schmitt/schmitt_vin_vout.png}
		\caption{}
	\end{subfigure}
	\begin{subfigure}{0.49\linewidth}
		\includegraphics[width = \linewidth]{immagini/schmitt/schmitt_vin_vout_rumoroso.png}
		\caption{}
	\end{subfigure}
	\\[0.5cm]
	\begin{subfigure}{0.49\linewidth}
		\includegraphics[width = \linewidth]{immagini/schmitt/schmitt_v+_vout.png}
		\caption{}
	\end{subfigure}
	\begin{subfigure}{0.49\linewidth}
		\includegraphics[width = \linewidth]{immagini/schmitt/schmitt_vin_v+.png}
		\caption{}
	\end{subfigure}
	\caption{Figure \textit{a} e \textit{b}: $V_{in}$ in giallo e $V_{out}$ in azzurro, confronto con segnale in ingresso non rumoroso (\textit{a}) e rumoroso (\textit{b}).
		Figura \textit{c}: in giallo la soglia dinamica, $V^+$, in azzurro l'uscita.
		Figura \textit{d}: Soglia dinamica confrontata con l'ingresso (in giallo).}
	\label{fig:schmitt_oscilloscopio}
\end{figure}

\FloatBarrier

\section*{Oscillatore}
L'oscillatore è un circuito elettronico che genera un segnale ad onda quadra, in questa particolare implementazione si sfrutta il \textit{Trigger di Schmitt} e la carica/scarica di un condensatore, figura \ref{fig:schematico_oscillatore}.
\begin{figure}
	\centering
	\includegraphics[width=0.4\linewidth]{immagini/ocillatore/circuito.png}
	\caption{Schematico oscillatore.}
	\label{fig:schematico_oscillatore}
\end{figure}
Il funzionamento del circuito si può suddividere in due fasi:
\begin{itemize}
    \item $V_{out} = V_{DD}$: La corrente fluisce da destra verso sinistra caricando il condensatore con una costante di tempo $\tau = R \cdot C$, inoltre si ottiene un potenziale nel morsetto positivo, dell'amplificatore, pari a:
    \begin{align*}
        V^+ = V^+_H \overset{\tiny\mathrm{Hp:}\,R_1 = R_2}{\longrightarrow} = \frac{V_{DD}}{2}
    \end{align*}
    Si rimane in questa condizione finché la capacità non si carica fino ad avere una caduta di tensione pari (o maggiore) a $V^+_H$.
    \item Nel momento in cui $V^-$ supera $V^+$ in uscita si ottiene $V_{SS}$, la quale modifica la $V^+$ portandola a una tensione pari a $V^+_L = \frac{V_{SS}}{2}$. In questa condizione la corrente in $R$ scorre da sinistra verso destra, scaricando il condensatore fino a quando la tensione in $V^-$ diventa inferiore a $V^+_L$ portando di nuovo il circuito nello stato precedente. 
\end{itemize}
Tramite l'equazioni di carica e scarica:
\begin{align*}
    V_c (t) = V^- (t) &= V_{\text{finale}} + (V_{\text{iniziale}} - V_{\text{finale}} ) \cdot e^{-\sfrac{t}{\tau}}
\end{align*}
Si può determinare i tempi un cui $V_{out}$ è alta e bassa, determinando così il valore della frequenza di oscillazione.
\begin{align*}
    \text{Carica}\,\, &\Rightarrow V_{DD} + (V^+_L - V_{DD}) \cdot e^{-\sfrac{T_1}{\tau}} = V^+_H\\
    \text{Scarica}\,\, &\Rightarrow V_{SS} + (V^+_H - V_{SS}) \cdot e^{-\sfrac{T_2}{\tau}} = V^+_L
\end{align*}
    Per le quali utilizzando $\left| V_{DD} \right| = \left| V_{SS} \right|$ si ottiene:
\begin{align*}
        T_1 = T_2 = \tau \ln\left( \frac{R_2 + 2R_1}{R_2} \right) 
\end{align*}
Questo risultato ci porta concludere che il duty cylce è pari al $50\%$, e la frequenza di oscillazione è pari a $ f = \frac{1}{T_1 + T_2}$, la quale e inversamente proporzionale alla costante di tempo $\tau$.

\noindent I valori utilizzati questo circuito sono indicati a tabella \ref{tab:oscillatore}, a figura \ref{fig:oscillatore} viene mostrato l'andamento dell'uscita al variare di $V^-$. Infine si sono effettuate diverse prove per confermare la relazione $f \propto \sfrac{1}{\tau}$ modificando la resistenza variabile $R$, figura \ref{fig:ocillatore_freq_tau}.

\noindent In particolare questo circuito, mantenendo sempre $\left| V_{DD} \right| = \left| V_{SS} \right|$, avrà sempre un duty cycle pari al $50\%$, se si volesse modificare si potrebbero utilizzare due resistenze diverse, una per la carica e una per la scarica, utilizzando due diodi per far scorrere la corrente nel modo corretto. 
\begin{table}[h]
	\centering
	\setlength{\tabcolsep}{20pt}
	\begin{tabular}{c c}
		\toprule
		Elemento & Valore             \\
		\midrule
		$V_{DD}$ & $10\,V$            \\
		$R_1$    & $9.0\,K\Omega$     \\
		$R_2$    & $9.0\,K\Omega$     \\
		$R$    & Resistenza variabile    \\
		$C$      & $70\,\mathrm{nF}$  \\
		\bottomrule
	\end{tabular}
	\caption{Valori utilizzati nell'oscillatore.}
	\label{tab:oscillatore}
\end{table}

\begin{figure}[h]
    \centering
    \includegraphics[width=0.6\linewidth]{immagini/ocillatore/oscillatore.PNG}
    \caption{Schermata dell'oscilloscopio, in giallo $V^-$ mentre in azzurro l'uscita $V_{out}$.}
    \label{fig:oscillatore}
\end{figure}

\begin{figure}[h]
	\centering
	\includegraphics[width=0.6\linewidth]{immagini/ocillatore/freq_tau.png}
	\caption{Variazione della frequenza al variare della costante di tempo $\tau$.}
	\label{fig:ocillatore_freq_tau}
\end{figure}
\FloatBarrier

\section*{Monostabile}
Il Monostabile è un circuito elettronico che riceve in ingresso un segnale e dà in uscita un impulso di durata ben definita.
Inoltre, il fronte d'onda ascendente del segnale in uscita è sincronizzato con il fronte d'onda discendente del segnale in ingresso.

\begin{figure}[h]
	\centering
	\includegraphics[width=0.6\linewidth]{immagini/monostabile/circuito.png}
	\caption{Schematico del circuito monostabile.}
	\label{fig:schematico_monostabile}
\end{figure}

In una configurazione iniziale con diodo collegato in parallelo alla capacità, questo porta l'uscita ad essere costante al valore di accensione del diodo circa 0.7V a causa del fluire della corrente nel percorso a resistenza più bassa e quindi la capacità smette di caricarsi.

La presenza del derivatore (filtro passa-alto) permette di estrarre gli impulsi a delta di dirac da un segnale in ingresso e questo fa si che la soglia dinamica cambi.
In aggiunta, collegando un diodo al derivatore fa si che passino solo gli impulsi negativi e quindi che il segnale venga disaccoppiato dal resto del circuito.

La durata dell'impulso è definita dalla costante tau e dalle resistenze R1 e R2:
formule...

Tabella dati
\begin{table}[h]
	\centering
	\setlength{\tabcolsep}{20pt}
	\begin{tabular}{c c}
		\toprule
		Elemento & Valore             \\
		\midrule
		$freq$   & $100\,\mathrm{Hz}$ \\
		$V_{DD}$ & $10\,V$            \\
		$R_1$    & $9.0\,K\Omega$     \\
		$R_2$    & $9.0\,K\Omega$     \\
		$R_3$    & $10.3\,K\Omega$    \\
		$R_4$    & $9.0\,K\Omega$     \\
		$C$      & $70\,\mathrm{nF}$  \\
		$C_T$    & $1\,\mathrm{nF}$   \\
		\bottomrule
	\end{tabular}
	\caption{Valori utilizzati nel circuito monostabile.}
	\label{tab:monostabile}
\end{table}


\end{document}

