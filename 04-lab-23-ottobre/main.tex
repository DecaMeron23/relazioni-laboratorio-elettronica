\documentclass[a4paper,12pt]{article}
\usepackage{graphicx}       %LaTeX package to import graphics

\usepackage[T1]{fontenc}
\usepackage[italian]{babel}

\usepackage{subcaption}

\usepackage{hyphenat}
\usepackage{array}
\usepackage{booktabs}       % Per linee orizzontali migliori
\usepackage{caption}        % Per personalizzare le didascalie<
\usepackage{multirow}       % Per combinare celle nelle colonne
\usepackage{float}
\usepackage{hyperref}

\usepackage{bm} 

\usepackage{amsmath}        % Per migliorare l'aspetto delle formule

\usepackage{todonotes}      % mettere le note dentro il documento

\usepackage{siunitx}        % Per formattare le unità di misura
\usepackage{gensymb}        % Simboli come °
\usepackage{xfrac}          % per fare le frazioni inclinate

\usepackage{pdfpages}

\usepackage{import}
\usepackage{frontespizio}

\usepackage{placeins}       %per non far andare le immagini al di fuori delle sezioni utilizzare il comando: \FloatBarrier non far superare le immagini quel punto

\usepackage{adjustbox}      % per dimensionare le immagini in modo automatico


\begin{document}

\includepdf{./frontespizio/frontespizio.pdf}

\section*{Inverter CMOS}
\begin{figure}[H]
	\centering
	\includegraphics[width=0.5\linewidth]{immagini/inverter/circuitoLogico1Led.png}
	\caption{Schematico inverter.}
	\label{fig:inverterStatica}
\end{figure}
Per la realizzazione del circuito mostrato in \autoref{fig:inverterStatica} è stato utilizzato il circuito integrato \textbf{CB4069}, tale dispositivo è dotato di 14 pin: il pin 14 è collegato alla tensione positiva di alimentazione, mentre il pin 7 è collegato a massa, tutti gli altri pin rappresentano gli ingressi e le uscite dei sei inverter integrati (da $A$ e $\overline{A}$ fino a $F$ e $\overline{F}$).
È stata impostata una \textbf{V\textsubscript{DD}} di \SI{5}{\volt} con l’obiettivo di verificare la \textbf{caratteristica statica} dell’inverter. 
Per farlo, si forza uno stato logico 0 o 1 utilizzando una resistenza in serie a un interruttore collegato all’ingresso dell’inverter. 
In uscita è stata posta una resistenza con in serie un diodo LED, in modo che con uscita logica 1 il LED risulti acceso e con uscita logica 0 risulti spento come mostrato in \autoref{fig:statica}.
Successivamente andando a modificare il circuito come in \autoref{fig:inverterDinamica}, è stato possibile verificare la \textbf{caratteristica dinamica} collegando l’ingresso del circuito a un generatore di forme d’onda che fornisce un segnale ad onda quadra con un’ampiezza di \SI{5}{\volt\text{-picco-picco}} e un offset di \SI{2.5}{\volt}, simulando la commutazione di un bit tra 0 e 1, per osservare la negazione in uscita. 
Infine, sono stati calcolati il \textbf{tempo di salita}, il \textbf{tempo di discesa} e il \textbf{ritardo di propagazione} dell’inverter. 
In \autoref{fig:ritardo_propagazione} è riportata la misura del ritardo di propagazione, pari a circa \SI{23.6}{\nano\second}. 
Le \autoref{fig:salita} e \autoref{fig:discesa} mostrano rispettivamente i fronti di salita e discesa del segnale in uscita, con tempi pari a circa \SI{30}{\nano\second} e \SI{20}{\nano\second}. 

\begin{figure}[H]
    \centering
    \begin{subfigure}[b]{0.45\textwidth}
        \centering
        \includegraphics[width=\textwidth]{immagini/inverter/on.png}
        \caption{Uscita logica 1 – LED acceso}
        \label{fig:led_on}
    \end{subfigure}
    \hfill
    \begin{subfigure}[b]{0.45\textwidth}
        \centering
        \includegraphics[width=\textwidth]{immagini/inverter/off.png}
        \caption{Uscita logica 0 – LED spento}
        \label{fig:led_off}
    \end{subfigure}
    \caption{Verifica della caratteristica statica dell'inverter tramite LED}
    \label{fig:statica}
\end{figure}

\begin{figure}[H]
    \centering
    
    \begin{subfigure}[b]{0.45\textwidth}
        \centering
        \includegraphics[width=\linewidth]{immagini/inverter/circuitoLogico1NoLed.png}
        \caption{Schematico dell'inverter per la verifica della caratteristica dinamica.}
        \label{fig:inverterDinamica}
    \end{subfigure}
    \hfill
    \begin{subfigure}[b]{0.45\textwidth}
        \centering
        \includegraphics[width=\linewidth]{immagini/inverter/TEK00099.PNG}
        \caption{Caratteristica dinamica dell'inverter.}
        \label{grafico}
    \end{subfigure}
    \caption{Verifica della caratteristica dinamica dell'inverter: a sinistra lo schema del circuito, a destra la risposta osservata all'oscilloscopio.}
    \label{fig:inverterDinamica}
\end{figure}

\begin{figure}[H]
    \centering
    \begin{subfigure}[b]{0.48\textwidth}
        \centering
        \includegraphics[width=\linewidth]{immagini/inverter/TEK00100.PNG}
        \caption{Ritardo di propagazione (\SI{23.6}{\nano\second}).}
        \label{fig:ritardo_propagazione}
    \end{subfigure}
    \hfill
    \begin{subfigure}[b]{0.48\textwidth}
        \centering
        \includegraphics[width=\linewidth]{immagini/inverter/TEK00101.PNG}
        \caption{Transizioni complete.}
        \label{fig:transizioni_complete}
    \end{subfigure}
    \\[1em]
    \begin{subfigure}[b]{0.48\textwidth}
        \centering
        \includegraphics[width=\linewidth]{immagini/inverter/TEK00102.PNG}
        \caption{Tempo di salita (\SI{30}{\nano\second}).}
        \label{fig:salita}
    \end{subfigure}
    \hfill
    \begin{subfigure}[b]{0.48\textwidth}
        \centering
        \includegraphics[width=\linewidth]{immagini/inverter/TEK00103.PNG}
        \caption{Tempo di discesa (\SI{20}{\nano\second}).}
        \label{fig:discesa}
    \end{subfigure}
    \caption{Misure sperimentali della caratteristica dinamica dell'inverter CMOS.}
    \label{fig:dinamica_misure}
\end{figure}

\section{Latch}

\end{document}

